\documentclass[beamer]{standalone}

\usetheme{naked}
\setbeamercolor{alerted text}{fg=green!50!black}
\setbeamercolor{box title}{fg=purple}
\setbeamertemplate{frametitle}{}

% \usepackage{xkeyval}
% \usepackage{pgffor}

\usepackage{standalone}

\makeatletter
../code-graphs.tex
\makeatother

% vertical centering of cells
% see http://tex.stackexchange.com/questions/46386/vertically-center-cells-of-a-table
\usepackage{array}% http://ctan.org/pkg/array
\newcolumntype{M}{>{\centering\arraybackslash}m{\dimexpr.17\linewidth-2\tabcolsep}}

% remove space between margin and lists
\usepackage{enumitem}
\setitemize{label=\usebeamerfont*{itemize item}%
  \usebeamercolor[fg]{itemize item}
  \usebeamertemplate{itemize item}}
\setlist{leftmargin=*,labelindent=0cm}

%\usepackage{amsmath}

\usepackage{tikz}
\usepackage{tkz-graph}
\usepackage{tkz-berge}
\usepackage{tkz-berge-add}

\usepackage[utf8]{inputenc}
% \usepackage{libertine}
% \usepackage[libertine]{newtxmath}

%\usepackage{lxfonts}

\usepackage{cabin}
\usepackage{mathastext}

\newcommand{\graphcaption}[4][gray!80!white]{\draw (#2,#3) node [fill=#1]{#4};}

\SetVertexSimple[FillColor=gray, MinSize=10pt, LineWidth=1.5pt]

\tikzset{EdgeStyle/.style= {%
    color           = white,
    double          = black,
    double distance = 2.5pt}}

\newcommand{\setof}[2]{\left\{\,#1\mid #2\,\right\}}

\newcommand{\triangulo}[4]{%
  \shadedraw[inner color=#4,opacity=0.8,line width=1pt]
  (#1.center) -- (#2.center) -- (#3.center) -- cycle;}

\newcommand{\triangleshaded}[3]{%
  \draw[fill=gray]
  (#1.center) -- (#2.center) -- (#3.center) -- cycle;}

\newcommand{\triang}[3]{%
  \shadedraw[inner color=gray,,opacity=0.8,line width=1pt]
  (#1.center) -- (#2.center) -- (#3.center) -- cycle;}

\begin{document}

\SetVertexSimple[FillColor=gray, MinSize=1pt, InnerSep=2pt, LineWidth=0.5pt]

\tikzset{EdgeStyle/.style= {%
    color           = white,
    double          = black,
    double distance = 0.5pt}}

\setlength{\fboxsep}{1pt}

\begin{standaloneframe}
  \scriptsize
  \begin{center}
    \foreach \n in {47,48,49}{%
      \begin{minipage}{0.3\linewidth}
        \centering
        \graphst{i=\n}{0.8}\\ \names{i=\n}
      \end{minipage}
    }
    \bigskip

    \foreach \n in {50,53,56}{%
      \begin{minipage}{0.3\linewidth}
        \centering
        \graphst{i=\n}{0.8}\\ \names{i=\n}
      \end{minipage}
    }
    \bigskip

    6 gráficas restantes
  \end{center}
  
  \pause

  \begin{itemize}
  \item Las gráficas \names{i=47}, \names{i=49}, \names{i=50} y
    \names{i=56} son tales que toda extensión de ellas es Helly, por
    resultado de Larrión, Pizaña, V. (2013): \textsl{Si $G$ es libre de
      $4,5,6$-ruedas, entonces $K(G)$ es Helly}.\pause
  \item Toda extensión de la gráfica \names{i=48} es divergente, por
    resultado de Larrión, Pizaña, V. (2014).\pause
  \item La gráfica \names{i=53} tiene una única extensión, con
    comportamiento desconocido.
  \end{itemize}
\end{standaloneframe}
\end{document}
