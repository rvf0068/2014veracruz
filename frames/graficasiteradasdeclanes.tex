\documentclass[beamer]{standalone}

\usetheme{naked}
\setbeamercolor{alerted text}{fg=green!50!black}
\setbeamercolor{box title}{fg=purple}
\setbeamertemplate{frametitle}{}

% \usepackage{xkeyval}
% \usepackage{pgffor}

\usepackage{standalone}

\makeatletter
../code-graphs.tex
\makeatother

% vertical centering of cells
% see http://tex.stackexchange.com/questions/46386/vertically-center-cells-of-a-table
\usepackage{array}% http://ctan.org/pkg/array
\newcolumntype{M}{>{\centering\arraybackslash}m{\dimexpr.17\linewidth-2\tabcolsep}}

% remove space between margin and lists
\usepackage{enumitem}
\setitemize{label=\usebeamerfont*{itemize item}%
  \usebeamercolor[fg]{itemize item}
  \usebeamertemplate{itemize item}}
\setlist{leftmargin=*,labelindent=0cm}

%\usepackage{amsmath}

\usepackage{tikz}
\usepackage{tkz-graph}
\usepackage{tkz-berge}
\usepackage{tkz-berge-add}

\usepackage[utf8]{inputenc}
% \usepackage{libertine}
% \usepackage[libertine]{newtxmath}

%\usepackage{lxfonts}

\usepackage{cabin}
\usepackage{mathastext}

\newcommand{\graphcaption}[4][gray!80!white]{\draw (#2,#3) node [fill=#1]{#4};}

\SetVertexSimple[FillColor=gray, MinSize=10pt, LineWidth=1.5pt]

\tikzset{EdgeStyle/.style= {%
    color           = white,
    double          = black,
    double distance = 2.5pt}}

\newcommand{\setof}[2]{\left\{\,#1\mid #2\,\right\}}

\newcommand{\triangulo}[4]{%
  \shadedraw[inner color=#4,opacity=0.8,line width=1pt]
  (#1.center) -- (#2.center) -- (#3.center) -- cycle;}

\newcommand{\triangleshaded}[3]{%
  \draw[fill=gray]
  (#1.center) -- (#2.center) -- (#3.center) -- cycle;}

\newcommand{\triang}[3]{%
  \shadedraw[inner color=gray,,opacity=0.8,line width=1pt]
  (#1.center) -- (#2.center) -- (#3.center) -- cycle;}

\begin{document}
\begin{standaloneframe}
  Las \alert{gráficas iteradas de clanes} se definen como:
  \begin{equation*}
    K^{0}(G)=G,\qquad K^{n}(G)=K\big(K^{n-1}(G)\big),\quad n\geq1.
  \end{equation*}
  \pause
  \begin{center}
    \begin{tikzpicture}
      \pgfmathsetmacro{\edgel}{1.8}
      \pgfmathsetmacro{\rad}{\edgel/sqrt(2)}
      \grHouse[RA=\rad]
      \Edge(a0)(a2)
      \Edge(a1)(a3)   
      \graphcaption{0}{-0.9*\edgel}{$G$}
      \uncover<3->{%
        \grPath[RA=\edgel,form=2,rotation=90,x=3,y=\edgel/2]{2}
        \graphcaption{3}{-0.9*\edgel}{$K(G)$}
      }
      \uncover<4->{%
        \Vertex[x=6,y=\edgel/2]{x}
        \graphcaption{6}{-0.9*\edgel}{$K^{2}(G)$}
      }
    \end{tikzpicture}
  \end{center}
\end{standaloneframe}
\end{document}
