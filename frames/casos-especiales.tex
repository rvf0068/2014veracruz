\documentclass[beamer]{standalone}

\usetheme{naked}
\setbeamercolor{alerted text}{fg=green!50!black}
\setbeamercolor{box title}{fg=purple}
\setbeamertemplate{frametitle}{}

% \usepackage{xkeyval}
% \usepackage{pgffor}

\usepackage{standalone}

\makeatletter
../code-graphs.tex
\makeatother

% vertical centering of cells
% see http://tex.stackexchange.com/questions/46386/vertically-center-cells-of-a-table
\usepackage{array}% http://ctan.org/pkg/array
\newcolumntype{M}{>{\centering\arraybackslash}m{\dimexpr.17\linewidth-2\tabcolsep}}

% remove space between margin and lists
\usepackage{enumitem}
\setitemize{label=\usebeamerfont*{itemize item}%
  \usebeamercolor[fg]{itemize item}
  \usebeamertemplate{itemize item}}
\setlist{leftmargin=*,labelindent=0cm}

%\usepackage{amsmath}

\usepackage{tikz}
\usepackage{tkz-graph}
\usepackage{tkz-berge}
\usepackage{tkz-berge-add}

\usepackage[utf8]{inputenc}
% \usepackage{libertine}
% \usepackage[libertine]{newtxmath}

%\usepackage{lxfonts}

\usepackage{cabin}
\usepackage{mathastext}

\newcommand{\graphcaption}[4][gray!80!white]{\draw (#2,#3) node [fill=#1]{#4};}

\SetVertexSimple[FillColor=gray, MinSize=10pt, LineWidth=1.5pt]

\tikzset{EdgeStyle/.style= {%
    color           = white,
    double          = black,
    double distance = 2.5pt}}

\newcommand{\setof}[2]{\left\{\,#1\mid #2\,\right\}}

\newcommand{\triangulo}[4]{%
  \shadedraw[inner color=#4,opacity=0.8,line width=1pt]
  (#1.center) -- (#2.center) -- (#3.center) -- cycle;}

\newcommand{\triangleshaded}[3]{%
  \draw[fill=gray]
  (#1.center) -- (#2.center) -- (#3.center) -- cycle;}

\newcommand{\triang}[3]{%
  \shadedraw[inner color=gray,,opacity=0.8,line width=1pt]
  (#1.center) -- (#2.center) -- (#3.center) -- cycle;}

\begin{document}

\SetVertexSimple[FillColor=gray, MinSize=0.7pt, InnerSep=0.7pt, LineWidth=0.5pt]

\tikzset{EdgeStyle/.style= {%
    color           = white,
    double          = black,
    double distance = 0.5pt}}

\setlength{\fboxsep}{1pt}

\begin{standaloneframe}
    \begin{center}
    %\tiny
    \scriptsize
    \begin{tabular}{|M|M|l|}
      \hline
      $L$ & loc. $L$ & Comportamiento\\
      \hline
      $K_{3}$ & 1 ($K_{4}$) & Nula\\
      $C_{4}$ & 1 ($O_{3}$) & Divergente (Neumann-Lara, 1976)\\
      $K_{4}$ & 1 ($K_{5}$) & Nula\\
      $C_{5}$ & 1 (icosaedro) & Divergente (Pizaña, 2003)\\
      \graphst{i=25}{0.3} & 1 ($\overline{C_{8}}$) & Divergente (Neumann-Lara, 1976)\\
      $K_{5}$ & 1 ($K_{6}$) & Nula\\
      $C_{6}$ & $\infty$ & Divergentes (Larrión, Neumann-Lara, 2000)\\
      \graphst{i=57}{0.3} & 1 ($L(O_{3})$) & Divergente ($K^{3}(L(O_{3}))=O_{8}$)\\
      \graphst{i=59}{0.3} & $\infty$ ($C_{n}^{2,3}$, $n\geq10$) & Autóclanas (Larrión, Neumann-Lara, 1997)\\
      \graphst{i=60}{0.3} & 1 ($C_{10}^{2,4}$) & Divergente ($K(G)=\mathrm{Susp}(C^{2}_{10})$, Neumann-Lara, 1976)\\
      $K_{3,3}$ & 1 ($K_{3,3,3}$) & Divergente (Neumann-Lara, 1976)\\
      \graphst{i=62}{0.2} & 1 ($\overline{C_{9}}$) & Divergente (Neumann-Lara, 1976)\\
      $O_{3}$ & 1 ($O_{4}$) & Divergente (Neumann-Lara, 1976)\\
      $K_{6}$ & 1 ($K_{7}$) & Nula\\
      \hline
    \end{tabular}

    \bigskip

    14 casos especiales
  \end{center}
\end{standaloneframe}
\end{document}
